\documentclass[11pt]{article}

\usepackage{homeworkpkg}
\usepackage{amsmath}
\usepackage{enumerate}

%% Local Macros and packages: add any of your own definitions here.

\begin{document}

% Homework number, your name and NetID
\homework
{1} %specify the number of the assignment here
{Jiahe Jiang (jiahej2)}

\section*{Q1}
\begin{enumerate}[a.]
	\item
		\textit{True}

		We can assume that outliers are at least one
		standerd deviation away from the mean.

		Let \(n\) be the number of data points, \(\mu\) be the sample mean
		\(\sigma\) be the sample standard deviation, and \(x\) be the outlier,
		where \(\|{x - \mu} \| \geq \sigma\).

		If we delete a outlier, then the new standard deviation will be

		\begin{align}
			\sigma'^2 - \sigma^2 & = \frac{1}{(n - 1)^2} \Bigl( n \sigma ^2  + n \mu ^2 - x ^2\Bigr) - \left( \frac{n \mu - x}{n- 1} \right)^2 - \sigma ^2 \\
								 & = \frac{1}{(n - 1)^2} \Bigl(n\bigl(\sigma^2 - (x - \mu)^2 - \sigma^2\bigr)\Bigr)                                        \\
								 & < 0
		\end{align}

		Therefore, the new standard deviation will be smaller than the original one.
	\item
		\textit{True}

		Let \(n = 2\), \(x_1 = 1\), \(x_2 = 3\), then
		\(\mu = 2\), \(\sigma^2 = 1\)

		Then \(z_1\) = \(\frac{1 - 2}{\sqrt{1}} < 0\)
	\item
		\textit{False}

		Consider two discrete distributions defined over a binary variable \( x \in \{0, 1\} \):

		\[
			p(0)=0.9,\quad p(1)=0.1,
		\]
		\[
			q(0)=0.5,\quad q(1)=0.5.
		\]

		\begin{align*}
			D_{\mathrm{KL}}(p \parallel q) &= 0.9 \ln\left(\frac{0.9}{0.5}\right) + 0.1 \ln\left(\frac{0.1}{0.5}\right)\\[1ex]
										   &\approx 0.3681.
		\end{align*}
		\begin{align*}
			D_{\mathrm{KL}}(q \parallel p) &= 0.5 \ln\left(\frac{0.5}{0.9}\right) + 0.5 \ln\left(\frac{0.5}{0.1}\right)\\[1ex]
										   &\approx 0.5108.
		\end{align*}
	\item
		\textit{False}

		\begin{align*}
			\rho_{X,Y} &= \frac{\operatorname{cov}(X,Y)}{\sigma_X \sigma_Y}\\
					  &= \frac{\mathbb{E}[(X - \mu_x)(Y - \mu_y)]}{\sigma_X \sigma_Y}\\
			\rho_{2X,Y} &= \frac{\operatorname{cov}(2X,Y)}{\sigma_{2X} \sigma_Y}\\
					  &= \frac{\mathbb{E}[(2X - 2\mu_x)(Y - \mu_y)]}{2\sigma_X \sigma_Y}\\
					  &= \frac{2\mathbb{E}[(X - \mu_x)(Y - \mu_y)]}{2\sigma_X \sigma_Y}\\
					  &= \rho_{X,Y}
		\end{align*}
	\item
		\textit{False}
		Outliers have greater effect on variances because squaring amplifies the impact of large differences. 
	\item
		\textit{False}
		Due to the curse of dimensionality, a dataset with high dimensionality is more likely to sparsity.
	\item
		\textit{True}
		Let \(\mathbf{x} = (x_1, x_2, \ldots, x_n)\) and \(\mathbf{y} = (y_1, y_2, \ldots, y_n)\) be two points in \(\mathbb{R}^n\). Define the difference vector
		\[
			\mathbf{z} = \mathbf{x} - \mathbf{y} = (z_1, z_2, \ldots, z_n),
		\]
		where \(z_i = x_i - y_i\) for \(i = 1, \ldots, n\).

		The Manhattan distance between \(\mathbf{x}\) and \(\mathbf{y}\) is given by:
		\[
			d_1(\mathbf{x}, \mathbf{y}) = \|\mathbf{z}\|_1 = \sum_{i=1}^n |z_i|.
		\]

		The Euclidean distance between \(\mathbf{x}\) and \(\mathbf{y}\) is given by:
		\[
			d_2(\mathbf{x}, \mathbf{y}) = \|\mathbf{z}\|_2 = \sqrt{\sum_{i=1}^n z_i^2}.
		\]

		For any vector \(\mathbf{z}\), we have
		\[
			\|\mathbf{z}\|_2 \leq \|\mathbf{z}\|_1.
		\]

		Therefore, \( d_1(\mathbf{x}, \mathbf{y}) \leq d_2(\mathbf{x}, \mathbf{y}) \).
	\item
		\textit{False}

		\[
			Q_1 \approx -0.6745 \quad \text{and} \quad Q_3 \approx 0.6745.
		\]
		Thus, the IQR is approximately:
		\[
			\text{IQR} \approx 0.6745 - (-0.6745) \approx 1.349.
		\]

		Two interquartile ranges of the median is given by:
		\[
			\left[ -2 \times 1.349, \, 2 \times 1.349 \right] = \left[ -2.698, \, 2.698 \right].
		\]

		Using standard normal distribution, we find:
		\[
			P(Z \le 2.698) \approx 0.9965 \quad \text{and} \quad P(Z \le -2.698) \approx 0.0035.
		\]
		Thus, the probability is approximately:
		\[
			0.9965 - 0.0035 = 0.9930 
		\]
		Which is different from 95\%.
	\item
		\textit{False}
		A nominal attribute do not have an inherent order.
	\item
		\textit{False}
		They are two different concepts.
	\item
		\textit{True}
		Outliers in a box plot are the data points that lie below \(Q_1 - 1.5
		\times \text{IQR}\) or above \(Q_3 + 1.5 \times \text{IQR}\).
	\item
		\textit{True}
		\[
			z = \frac{x - \mu}{\sigma}
		\]
		if \(x < \mu\) then \(z < 0\)
\end{enumerate}
\clearpage

\section{Q2}
\begin{enumerate}[a.]
	\item \textit{answers for Q2}

	\item
\end{enumerate}

\clearpage
%uncomment the following line if you have any citations, and put the bibtex into citations.bib
% \bibliography{citations}
\bibliographystyle{ims}
\end{document}
